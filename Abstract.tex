The purpose of this Request for Proposal is to seek a solution to the
problem of chronic back pain among taxi drivers in the City of Toronto in
order to improve their quality of life. By mitigating or delaying the
onset of back pain, a greater number of taxi drivers will work without 
experiencing pain, and by extension discomfort. Minimizing discomfort of taxi drivers
will augment the quality of life of the community. 

Prolonged operation of a motor vehicle causes chronic pain, and taxi
drivers work shifts of between 8 and 12 hours. This results in a
greater frequency of lumbago pain among urban
taxi drivers as compared to the general population. The effects of chronic back pain
on taxi drivers could be significantly improved through augmenting 
the ergonomics of the work environment \cite{LostWorkdays}.
%missing source on the fact that ergonomics improves the effects of back pain
%the quote used was "the impact of back pain can be reduce through intervetion strategies such as training, job redesign, work enviroment engineering..."

All designs must succeed in reducing both severity and frequency of
chronic pain among taxi drivers, and achieve this at a minimum cost to
the stakeholders. Further, these designs must comply with current
Canadian Motor Vehicle Regulations, install in existing taxis, and
must not reduce available passenger space or hinder ingress and
egress from the vehicle. Solutions are more specifically targeted
towards low-cost additions to pre-existing car seats to mitigate lower
back pain.
