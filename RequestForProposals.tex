\documentclass[11pt]{article}
\usepackage[super]{natbib}
\usepackage[top=1in, bottom=1in, left=1in, right=1in]{geometry}
\usepackage{textcomp}
\frenchspacing

\begin{document}
\title{Request for Proposals}
\author{F2139-01}
\maketitle

\section{Introduction}
\label{sec:intro}
This Request for Proposal seeks a solution to the problem of chronic
pain among taxi drivers in the City of Toronto. Research demonstrates a
link between long periods of driving and chronic pain\cite{KneePain, Okunribido2008}, and taxi drivers
are particularly susceptible to this due to their long shifts\cite{thestar2012}.
This document will begin by defining the key terms and the community.
Using this context, it will then establish the problem and provide existing
solutions to similar problems before providing an engineering framing for the problem.
\section{Definitions}
\label{sec:defs}
The definitions of ergonomics, quality of life, and chronic pain, as
well as their prevalence to taxi drivers are outlined in this section.

\subsection{Ergonomics}
\label{sec:ergonomics}

The International Ergonomics Association defines ergonomics as “the
study of human interactions and other elements within a system”. By
utilizing ergonomics, engineers are able to optimize the suitability
of equipment for the structure and functioning method of the human
body. The development of ergonomic- equipment is applicable to
taxi drivers because by optimizing the interaction of the cab driver
and the vehicle to avoid repetitive strain injuries, the onset of
chronic pain will be prolonged.

\subsection{Chronic Pain}
\label{sec:pain}

Continuous pain that persists for 12 weeks or more is defined to be
chronic pain. As taxi driver experience a gradual onset of chronic
pain, avoiding or delaying its onset would enhance their work
environment and furthermore the quality of life of a taxi driver.

\subsection{Quality of Life}
\label{sec:lifequal}

Quality of life is defined as an individual’s evaluation, based on
their society’s culture and value systems, of their current status in
life with regards to personal goals, expectations, standards and
concerns. Improving to the quality of life of taxi drivers is the
purpose of this report.

\section{Community Description}
\label{sec:community}
The taxicab community in the City of Toronto consists of over 10400 registered drivers, 
and 4849 issued legal license plates.To become a taxicab driver the following is required 
by the City of Toronto Municipal Licensing and Standards:
\begin{enumerate}
\item Complete the Effective Taxicab Training Program, a 17-day program run by the city. 
Minimum pass average is 80\%. Program includes certification in first aid and CPR.
\item Pass a criminal record background check
\item Have a G License
\item Pass a physical checkup by a qualified physician
\end{enumerate}
The total cost of this program is \$601.23\cite{MLS2013}.

Licences must renewed yearly on the anniversary of issue and a refresher course must 
be completed once every three years\cite{MLSChp545}.

iTaxiworkers association is the only representative body of the taxicab community. It
works to improve the quality of life of taxidrivers in Ontario by "providing legal defence
and political advocacy for justice and reform"\cite{iTaxiWorkers2012}. The Toronto chapter
has a membership of over 1000 people. Membership is \$35 per month\cite{iTaxi2012, Abdiemail}.
\subsection{Demographics}
The 2006 census conducted by iTaxi yielded the following results\cite{iTaxiWorkers2012}:
\begin{table}[h]
\centering
\caption{Basic demographic data for taxi drivers in Toronto}
\begin{tabular}{l c}
 Total Drivers & 11 055 \\
 Male & 96\% \\
 Female & 4\% \\
 Immigrants & 81\% \\
 English as a foreign language & 76\% \\
 Average hours worked per week & 48.1 \\
 Median income & \$11 949
\end{tabular}
\end{table}
\subsection{Plates}
There are a number of different types of plates issued by the city but the vast majority 
consist of two two types of plates: standard (3451) and ambassador (1313). 
Standard plates enable multiple people to share the same vehicle and can be operated 24/7. 
Ambassador plate can only be operated by one person (the owner of the plate) and can be 
operated for a maximum of 12 hours a day. An ambassador license plate is required to 
operate from Pearson International Airport. Standard licence plates can also be sold. 
The current market value is up to \$300 000. The average taxi drives between 100 000 and 
250 000 km per year. As a result the vehicles on average do not last more than 5 years. 
Payment is as follows: the drop rate is \$4.25 + 25 cents for every 0.143 km of 
travel\cite{thestar2012}.

\subsection{Ownership and Responsibility}

Almost all drivers belong to one of the major companies, like Beck, Diamond, Royal, City, 
Co-op etc. The driver pays a yearly fee (varies by company) for a dispatch radio. The 
driver is guided to a pickup location by a coordinator. Ownership also varies. For example,
 at Beck Taxi, most of the vehicles are owned by the drivers themselves who just pay the 
rate for the radio and get the car painted the company colours. At Royal taxi, most of the 
cars are owned by garages and leased by drivers. Other circumstances are a collective of 
drivers who share the price of a plate and a vehicle and operate shifts. In all 
circumstances the driver keeps the profit he/she makes during the shift, but is also 
directly responsible for the cleanliness and integrity of the vehicle, including gasoline. 
Inspections are made weekly, bi-weekly or monthly depending on the company and offenders 
have their frequency cut off and do not receive guidance to customers. Drivers will also 
tell other drivers if their cab is messy and report them if problem is not rectified. To 
be cut-off is horribly detrimental to drivers because if not guided from one pickup to 
the next, much time is wasted waiting for customers and as a result the cost of the radio 
and/or lease cannot be earned back and no profit is made\cite{thestar2012, Gowder2013}.

\section{Stakeholders}
\label{sec:stake}
Taxi drivers as a community are connected with many groups in the City
of Toronto. Any design for Taxi drivers may potentially impact the
stakeholders connected to Taxi Drivers as well.
 
\subsection{Taxi Drivers}
The primary stakeholder is the Taxi Driver community in the City of
Toronto. These taxi drivers constitute the defined community with a
need. Improving the quality of life of this group through mitigating
chronic pain is the main objective of this design.
 
\subsection{Taxi Driver Union: iTaxiworkers Association}
The association represents taxi drivers in Toronto and provides
support for conflicts arising from police, dispatchers, systems or
taxi companies. They are politically active in lobbying for improved
working conditions for Taxi Drivers. The successful development of a
design solution may involved dialogue with the association and the
successfully implementation may involve promotion by the iTaxiworkers
Association\cite{itaxi}.
 
\subsection{Customers}
Taxi drivers are employees of the service industry and the customers
as the clients of taxi drivers are the consumers of the services. An
ideal solution to improve the health of taxi drivers may require the
assistance of customers. An active solution to the proposed problem
may raise the safety and riding environment of the passenger.
 
\subsection{Taxi Manufacturers}
Taxi manufacturers provide vehicles and equipment for taxi drivers. A
technical solution requiring physical changes to the car would need
the support and approval of the manufacturers of taxis or taxi
parts. Said solution would also need to be achievable by technologies
available to the taxi manufacturers.
 
\subsection{Taxi Companies}
In the City of Toronto, the main taxi companies include Beck Taxi,
Diamond, Royal Taxi and Co-op Cabs \cite{abouttaxis}. These companies
may employ of taxi drivers or centralize dispatch operation for taxi
drivers. Changes in the operation or the physical nature of the
vehicle may require additional funding or approval from the taxi
companies.
 
\subsection{City of Toronto}
The City of Toronto standardizes taxi training, licenses taxi drivers
and establishes relevant policies affecting taxi drivers
\cite{CityofToronto}. Design changes to the operation of physical
nature of taxis are only possible with the approval of the city and
may yield greater success with endorsement by the city.
\section{Needs}
\label{sec:needs}
The workplace environment of taxi imposes strains on driver as it
requires repetitive motions (pressing a pedal), a fixed postural
position for extended periods of time, and the operation of a
vehicle. These requirements not only induce the gradual onset of lower
back, neck and knee pain in drivers, but also amplify it
\cite{POSTULATED}. The quality of life of a taxi driver is diminished
by the onset of pain; it becomes increasing more difficult to perform
necessary functions for their job. By reducing the amount of pain
experienced by taxi drivers, the standard of their job, and
furthermore their quality of life, would improve as driver would spend
less time working in conjunction with pain. Evidently, the onset of
neck, lower back as well as knee pain is a problem in taxi drivers,
and overcoming this issue would enhance their quality of life.
 
Compelling evidence not-only illustrates the elevated risk of urban
taxi drivers experiencing lower back pain, but also why it’s induced
by driving. Whole body vibration (WBV) is used to define the
mechanical energy oscillations transferred to the body as a whole
(contrast to specific regions in the body) [ppt]. In the instance of
taxi drivers, mechanical energy is transferred to them through the car
seat [ppt]. The most common and hazardous vibratory exposure,
regarding back problems, in occupational life is seated WBV
\cite{ODrivers@Risk}. Bio-mechanical studies reinforce the link
between LBP and vibration exposures. Studies support that taxi drivers
experience whole-body vibrations \cite{KneePain, Serious}. Lower back
pain is also stimulated by prolonged sitting [find source]. Thus, a
not-only a greater amount time driving, but also distance induces a
higher incidence rate of back injuries\cite{Question?}. This is
reinforced by a positive correlation between driving a vehicle and
patients experiencing LBP \cite{ODrivers@Risk}.  The study also
mentioned that the “confined space within taxicabs may put taxi
drivers at a greater risk for LBP, as bio-mechanical studies have
shown that the driving activities within automobiles can impose
postural strains on lumbar spines” \cite{KneePain}. The restrictive
environment of taxi drivers clearly aggravates their back pain. Other
work-related stresses in taxi drivers such as bending/twisting
movements, air pollution, violence, and psychological strains may
enhance the development of LBP \cite{KneePain, POSTULATED}. The
prevalence of LBP in taxi drivers is reinforced by statistics; in the
United States of America, 68.4 out of 100,000 (taxi and chauffeur)
drivers compared to the 35.4 average experience musculoskeletal
disorders \cite{68}. Also, low back pain is more prevalent in
Norwegian taxi drivers (66\% for women, and 59\% for men) compared to
the reference population (58\% for women, and 51\% for men). Similar
results were established for Taiwanese urban taxi drivers, “51\%
reported low back pain relative to 33\% for other professional
drivers” \cite{Question?} LBP in taxi drivers is prevalent and induced
by the whole body vibrations, prolonged sitting, and a restricted
range of motion, all of which are caused by the taxi workplace
environment.
 
Taxi drivers experience elevated levels of knee pain, resulting mainly
from excessive use of the knee joint in order to perform their
service. Elevated level of knee pain in taxi driver is illustrated by
its higher occurrence in drivers in Norway versus the local community
(29\% vs 25\%) \cite{KneePain}. This is reinforced by a Taiwanese
study that noted greater amount of knee pain in drivers than in its
national average \cite{KneePain}. Repetitive strain injury, defined as
an injury stimulated in a certain part of the body due to its overuse,
is the root of knee pain in drivers\cite{RSI}. As drivers constantly
use their knee to operate a vehicle, the joint is overused over time,
resulting in the onset of pain. This is supported by data suggesting
that “prolonged or repeated knee bending is a risk factor for knee OA
[osteoarthritis], and that risk may be higher in jobs which entail
both knee bending and mechanical loading”
\cite{Osteoarthritis}. Furthermore, a positive correlation between
repetitive motions during driving and knee pain has also been
found. Prior to 1990, 25\% of drivers experienced knee pain compared
with 18\% who operated vehicles made after 1990 \cite{KneePain}. This
downfall coincides with a drastic shift of automatic transmission
instead of manual, which requires more repetitive motion in the knee
joint \cite{KneePain}. This data illustrates that a greater amount of
pedal-pressing aggravates and elevates the risk of knee pain. It is
evident that greater amounts of driving, which provoke repetitive
strain injuries, induces knee pain.
 
Prolonged sitting and twisting of the trunk are requirements of the
taxi driver occupation, and both of which directly stimulate neck
pain. The Canadian Centre for Occupational Health and Safety believes
that driving can cause pain in the neck\cite{proof}. A study concluded
that prolonged sitting and twisting or bending of the trunk both
exhibit a positive correlation with neck pain \cite{neck}. As a taxi
driver’s occupation requires prolonged sitting, it is evident that
their work-place environment stimulates neck pain.
\section{Reference Designs}
\label{sec:designs}

\subsection{Problems with sitting}
\label{sec:sittingproblems}

%I'm not sure where this is meant to go, but it should go somewhere):**!!!!
Sitting for extended periods of time is a serious stress on the human body. As weight is adjusted
onto the thighs, body weight compresses the main arteries beneath the thighs. This results in 
decrease of blood flow. Muscles must do work to maintain an upright posture, and due to the decrease 
of blood flow, there is a buildup of lactic acids, which results in muscle spasms, also known as static
loading, and eventual muscle fatigue. Furthermore, all weight above the waist is balanced on the ischial 
tuberosities (in the buttocks), which are essentially inverted pyramids onto which all the weight is 
balanced. This results in discomfort. Moreover; the geometry of most seats results in a flattening of the 
spine. The lower spine in equilibrium position curves inwards towards the stomach. When curvature of the 
spine is flattened, the lumbar discs of the spine compress together which in turn results in a migration of 
the gel that insulates each disc. In the absence of cushioning gel, degradation of the disc occurs, which 
ultimately results in aggravation of the nerves (back pain), hernia and eventually rupture. The latter two 
require surgery\cite{ergoCentricAnatomy2011}.

\subsection{What constitutes a good seat?}
A good seat should have a pneumatic seat adjustment that acts as a shock absorber for the spine. The seat 
should allow a person to sit upright against the backrest with full contact while still leaving at least a 
three finger width clearance between the back of the knees and the seat. The shorter the seat pan, the better. 
The seat should waterfall downwards to reduce pressure on the veins beneath the thighs. The seat should also curve 
upwards at the sides to redistribute weight across a larger surface area. The seat itself should be neither dish 
up or down in the centre, because up results in pressure on the spine and down results in pressure on the veins. 
A faint ridge to match the geometry of the buttocks is optimal\cite{ergoCentricchair2011, Natpost2005}.


\subsection{An example of a good seat: ObusForme}
The seat has firm lumbar support that pushes the spine back into its natural position. 
It also has lateral curves that fit the geometry of the upper back to reduce muscle work\cite{ObusFormebackrest}. 
The backrest angle relative to the seat is just over 90\textdegree. The support also has a neck brace which supports
the neck, keeping the head aligned with the spine\cite{ObusFormedriverchair}.
ObusForme also has driver seats that reduce pressure on the veins and redistributes weight 
on both the spine and the ischial tuberosities by shaping to the human body\cite{ObusFormecushion}.

\subsection{Bose truck driver seat}
Truck drivers drive for up to 12 hours per day and as a result face many of the same issues as taxi drivers. 
Primarily is the issue of vibration and the effect on the spine. the Bose seat actively minimizes vibration 
by detecting vertical motion, even 0.25mm and up to 1000 Hz. The computer process the wave information and 
responds with a signal to an actuator that operates a pneumatic pump to raise or lower the seat to counter 
the incoming shock. Lowers vibration to between 0.2-0.3 eVDV (estimated vibration dose value). Normal is 
between 0.8-1.2 eVDV. Seat also shapes to back in the same way the ObusForme car seat\cite{Bosetruckseat}.

\subsection{Bus driver seat}
The following report was compiled by the Optimal Performance Consultants company regarding the 
inadequacies of a public bus cab and the negative effects on the health of the driver who has to sit 
for many hours a day, similar to a cab driver. Among suggestions to alter the seat to be more ergonomic, 
they also suggested a reform of the pedal operation that can result in long term ankle pain. 
Their suggestions were:
\begin{enumerate}
\item Pedals should not require a rotation of more than 25 degrees relative to the rest position. 
\item The more frequent the operation of the pedal, the smaller the minimum force limit should be, 
i.e. more sensitive pedals. Minimum lower bound is 40N of force, and the supremum is 400N.
\item The pedals should be mounted on an incline as opposed to flat against the floor to allow the leg 
muscles to exert most of the force, minimizing strain on the ankles. 
\item Most important was the suggestion to implement pedals to signal change of lane, thereby removing 
the wrist operation procedure and wear on the wrists\cite{Ismail2003}. 
\end{enumerate}



\section{Engineering Problems}
\label{sec:engprob}
Several solutions have been presented to the issues of chronic pain
(see section~\ref{sec:designs}), but implementing these solution in the context
of a taxi raises problems, which will be examined in this section. These 
problems are space, ease of use, and installation.

\subsection{Space}
\label{sec:space}

Solutions exist that are designed to function as desk chairs or seats for truck
drivers. These solutions, however, would need to be altered to fit in
a taxi. The CVG GSX 3000 truck seat, for example, measures 1113~mm from base
to top, and is 522~mm wide. The dimensions of a mid-size automobile
are more restrictive than a truck, so solutions will have to deal with
this reduced space.

\subsection{Ease of Use}
\label{sec:ease}
Taxi drivers must frequently enter and exit their vehicles in the course of
their duties, whether to assist customers with bags, or simply while awaiting
a customer. Compared to long-haul truck drivers, taxi drivers must have less
difficulty entering and exiting their vehicles. This should be kept in mind
while designing any aspect of the driver's area.
\subsection{Installation}
\label{sec:installation}

The vast majority of taxis are modified production cars, % Source: Interviews?
and so certain aspects of existing reference designs would be difficult to
implement. The standard steering wheel of an automobile, for example,
would be difficult to modify to a 60\textdegree~range of
rotation\cite{Ismail2003}. Feasibility of installation must then
be considered when assessing any implementation of a solution.

\section{Design Philosophy}
\label{sec:philo}
The design philosophy will guide the beginning stages of developing a
solution for the health needs of Taxi Drivers in the City of
Toronto. There are three key principles as high level objectives that
will guide the design process of the solution to the outlined problem.
 
\subsection{Efficacy}
The design must strive to alleviate the professional exposure to
chronic pain that taxi drivers experience. The main reason taxi
drivers consider a career change is health. Thus, it is crucial that
any design focuses on efficacy as the main priority\cite{chicago}. The
degree to which the persistent pain of taxi drivers is reduced is the
indicator of effectiveness. The ultimate goal would be to completely
eliminate pain and create comfort for taxi drivers in a working
environment.
 
\subsection{Prevention}
In addition to providing relief to drivers currently suffering from
chronic pains, the design would also seek to prevent future long
health problems. For new drivers, a preventative design would
eliminate the risk of chronic pain due to the occupation. While for
experienced drivers, prevention entails assurance that the chronic
pain reduced does not return.
 
\subsection{Accessibility}
To suit the Taxi driver community, accessibility both economically and
usage wise is important to accommodate. Considering that a majority of
taxi drivers are experience economic
uncertainty\cite{facey2003health}, any design aimed to improve the
quality of life in taxi drivers must economically accessible design to
the income of a taxi driver.
 
\subsection{Convenience}
For Taxi drivers, every minute could translate into possible business
time\cite{facey2003health}. The time consumed by taxi drivers to
operate any additional device should not deter from their
incomes. Thus designs should also seek to allow the achievement of an
elevated level of health without additional time consuming efforts.
\section{Project Requirements}
\label{sec:requirements}
This section will present the objectives, constraints, and criteria 
that frame the problem.

\subsection{High-level Objectives}
\label{sec:high-level-objectives}

\begin{enumerate}
\item Reduce the \emph{frequency} of chronic pain among Toronto taxi drivers
\item Reduce the \emph{severity} of chronic pain among Toronto taxi drivers
\item Achieve both of the above at a minimum cost to the stakeholders
\end{enumerate}

\subsection{Constraints}
\begin{enumerate}
\item \emph{Must} be capable of being installed in existing taxis.
\item \emph{Must} comply with all of the Canadian Motor Vehicle Safety Regulations,
specifically Schedule~IV \cite{motorregs}.
\item \emph{Must not} reduce the amount of available passenger space.
\item \emph{Must not} hinder ingress and egress from the vehicle.
\end{enumerate}
\subsection{Criteria}
The following criteria will be used to assess all designs.
\begin{table}[h]
\centering
\caption{Criteria and corresponding metrics}
\begin{tabular}{c p{10cm} }
  Criteria & Metric \\ \hline
  Cost & Combined manufacturing and installation price (lower is better) \\
  Ease of installation & Time required to install in taxi (less is better) \\
  Breadth of effectiveness & Percentile range accommodated by device (higher is better) \\
  Durability & Time before required maintenance (more is better) \\
  Severity & Level of reported pain according to test (lower is better) \\
  Frequency & Number of drivers reporting chronic pain (lower is better)
\end{tabular}
\end{table}

\bibliographystyle{IEEEtranN}
\bibliography{references,sourcessam,sourcesjudy,sourcesphil,sourcessam2}

\end{document}

