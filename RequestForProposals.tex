\documentclass[11pt]{article}
\usepackage[super]{natbib}
\usepackage[top=1in, bottom=1in, left=1in, right=1in]{geometry}
\usepackage{textcomp}
\usepackage{multirow}
\usepackage{graphicx}
\frenchspacing

\begin{document}
\title{Request for Proposal}
\author{F2139-01}
\maketitle
\begin{abstract}
The purpose of this Request for Proposal is to seek a solution to the
problem of chronic back pain among taxi drivers in the City of Toronto in
order to improve their quality of life. By mitigating or delaying the
onset of back pain, a greater number of taxi drivers will work without 
experiencing pain, and by extension discomfort. Minimizing discomfort of taxi drivers
will augment the quality of life of the community. 

Prolonged operation of a motor vehicle causes chronic pain, and taxi
drivers work shifts of between 8 and 12 hours. This results in a
greater frequency of lumbago pain among urban
taxi drivers as compared to the general population. The effects of chronic back pain
on taxi drivers could be significantly improved through augmenting 
the ergonomics of the work environment \cite{LostWorkdays}.
%missing source on the fact that ergonomics improves the effects of back pain
%the quote used was "the impact of back pain can be reduce through intervetion strategies such as training, job redesign, work enviroment engineering..."

All designs must succeed in reducing both severity and frequency of
chronic pain among taxi drivers, and achieve this at a minimum cost to
the stakeholders. Further, these designs must comply with current
Canadian Motor Vehicle Regulations, install in existing taxis, and
must not reduce available passenger space or hinder ingress and
egress from the vehicle. Solutions are more specifically targeted
towards low-cost additions to pre-existing car seats to mitigate lower
back pain.

\end{abstract}
\section{Introduction}
\label{sec:intro}
This Request for Proposal seeks a solution to the problem of chronic
pain among taxi drivers in the City of Toronto. Research demonstrates a
link between long periods of driving and chronic pain\cite{KneePain,
Okunribido2008}, and taxi drivers are particularly susceptible to this
due to their long shifts\cite{thestar2012, KneePain}. This document will introduce
definitions of key terms associated with the issue and the taxi driver community
in the City of Toronto. Based on the established context, the needs of taxi drivers
will be explored and framed into an engineering-oriented problem. 

\section{Definitions}
\label{sec:defs}
The definitions of ergonomics, chronic pain, needs, quality of life, as
well as their relation to taxi drivers are outlined in this section.

\subsection{Ergonomics}
\label{sec:ergonomics}

The International Ergonomics Association defines ergonomics as “the
study of human interactions and other elements within a system”\cite{ergo}. 
Through the utiliztion of ergonomics, engineers are able to optimize the suitability
of equipment for the structure and functioning method of the human
body. The development of ergonomic equipment is applicable to
taxi drivers because optimizing the interaction of the cab driver
and the vehicle to avoid repetitive strain injuries will postpone
the onset of chronic pain. 

\subsection{Chronic Pain}
\label{sec:pain}

Continuous pain that persists for 12 weeks or more is defined to be
chronic pain\cite{chronic}. As taxi drivers experience a gradual onset of chronic
pain, avoiding or delaying its onset would not only enhance their work
environment, but also their quality of life.

\subsection{Quality of Life}
\label{sec:lifequal}

Quality of life is defined as an individual's evaluation, based on
their society's culture and value systems, of their current status in
life with regards to personal goals, expectations, standards and
concerns\cite{WHO}. Improving to the quality of life of taxi drivers is the
purpose of this report.

\section{Community Description}
\label{sec:community}
The taxicab community in the City of Toronto consists of over 10400 registered drivers, 
and 4849 issued legal license plates. This section examines and 
illustrates certain characteristics of the taxicab community of Toronto such as economics, 
demographics, representation and accountability.

\subsection{Pre-requisites}
To become a taxicab driver the following is required by the City of Toronto Municipal Licensing and Standards:
\begin{enumerate}
\item Complete the \emph{Effective Taxicab Training Program}, a 17-day program run by the city. 
Minimum pass average is 80\%. Program includes certification in first aid and CPR.
\item Letter of clearance for a criminal record background check by the Toronto Police Service
\item G Class License accredited by Ministry of Transportation of Ontario 
\item Completion of a successful physical exam by a physician certified by the Royal College of Physicians and Surgeons 
\end{enumerate}
The total cost of this program is \$601.23\cite{MLS2013}.

Licences must be renewed anually %on the anniversary of issue 
and a refresher course must be completed once every three years\cite{MLSChp545}.

%This should not be here??
The iTaxiworkers association is the only representative body of the taxi driver community in the City of Toronto. 
The organization works to improve the quality of life of taxi drivers in Ontario by "providing legal defence
and political advocacy for justice and reform"\cite{iTaxiWorkers2012}. The Toronto chapter
has a membership of over 1000 people. Membership is \$35 per month\cite{iTaxi2012, Abdiemail}.

\subsection{Demographics}
The 2006 census conducted by iTaxiworkers yielded the following results\cite{iTaxiWorkers2012}:
\begin{table}[h]
\centering
\caption{Basic demographic data for taxi drivers in Toronto}
\begin{tabular}{l c}
 Total Drivers & 11 055 \\
 Male & 96\% \\
 Female & 4\% \\
 Immigrants & 81\% \\
 English as a foreign language & 76\% \\
 Average hours worked per week & 48.1 \\
 Median income & \$11 949 \\
\end{tabular}
\end{table}

Taxi drivers earn a portion of the standard rate of \$4.25 + 25 cents for every 0.143 km of 
travel\cite{thestar2012} and any tips given by the customers. Due to their unsteady flow of 
customer during working hours, taxi drivers and their families are often of low income since these 
drivers are almost always the sole breadwinner for their families\cite{Abdiphone}.

\subsection{Plates}
All taxi vechicles require a plate for legal operation. There are a number of different types of plates issued by the city. However, the vast majority 
consist of two types of plates: standard (3451) and ambassador (1313)\cite{thestar2012}. 
Ambassador plate can only be operated by one person (the owner of the plate) and can be 
operated for a maximum of 12 hours a day. An ambassador license plate is required to 
operate from Pearson International Airport. Standard plates enable multiple people to share the same vehicle
and can be operated 24/7. Standard licence plates can be sold, and their
current market value is up to \$300 000\cite{thestar2012}. On average, taxis travel between 100 000 and 
250 000 km per year\cite{thestar2012}, and last less than 5 years\cite{thestar2012}. 


\subsection{Ownership and Responsibility}

Almost all drivers belong to one of the major companies: Beck, Diamond, Royal, City and
Co-op. The driver pays a yearly fee (varies by company) for a dispatch radio which is used to guide the 
driver to a pickup location by a coordinator. Ownership also varies by company. For example, 
at Beck Taxi, most of the vehicles are owned by the drivers themselves who just pay the 
rate for the radio and get the car painted the company colours. At Royal taxi, most of the 
cars are owned by garages and leased to drivers. It is also common for a collection of 
drivers share the price of a plate as well as a vehicle and operate in shifts. Regardless of   
ownership status, drivers are directly responsible for the cleanliness and integrity of the vehicle as well as gasoline. 
Inspections by the company are made weekly, bi-weekly or monthly depending on the company. 
Failure to meeting a company's standards may result the termination of dispatcher services. Drivers receive 
warnings if their cab is left messy, and may be reported if the problem perpetuates. Being removed from
a company's service is detrimental to drivers because having guidance to a pickup location
is more profitable than searching for customers. The addition income through using a dispatcher 
service more than covers the cost of the radio and/or lease \cite{thestar2012, Gowder2013}.

\section{Stakeholders}
\label{sec:stake}
Taxi drivers as a community are connected with many groups in the City
of Toronto. Any new designs for taxi drivers may impact
stakeholders connected with their community.
 
\subsection{Taxi Drivers}
The primary stakeholder is the taxi driver community in the City of
Toronto. These taxi drivers constitute the previously-defined community with a
need. Improving the quality of life of this group through mitigating
chronic pain is the main objective of this design.
 
\subsection{Secondary Stakeholders} 
Customers and the taxi driver union directly interact with taxi 
drivers in the City of Toronto. These stakeholders would likely be 
affected by any changes in the design of taxis or the operation of taxis 
in the city. 

\subsubsection{Taxi Driver Union: iTaxiworkers Association}
The iTaxiworkers association represents taxi drivers in Toronto and provides 
support for conflicts arising between drivers, police, dispatchers, and 
taxi companies. They are politically active in lobbying for improved 
working conditions for taxi drivers. The successful development of a 
design solution may involve dialogue with the association and
successful implementation of a solution may involve promotion by the 
iTaxiworkers Association\cite{itaxi}. 

%fix that last sentence

\subsubsection{Customers}
Taxi drivers are part of the service and tourism industry. Customers,
clients of taxi drivers, are the consumers of the services. Taxi 
drivers in Toronto interact with both local and international customers 
24 hours a day. Solutions to improve the health of taxi drivers 
may require the assistance of customers or a change in the customer 
experience. An effective solution to the proposed problem may increase 
the safety of and improve the riding environment of passengers. 
 
 %take out the last sentence
 
\subsection{Additional Stakeholders}

The following stakeholders are associated with taxi drivers in the City 
of Toronto, but less directly related to their needs. 

\begin{table}[h]
  \centering
  \caption{Additional stakeholders}
    \begin{tabular}{ l p{5cm} p{5cm}}
    Stakeholder & Relations &	Interest \\ \hline
    Taxi Manufacturers & Provide vehicles and equipment for taxi drivers	& 
    Any technical solution needs to be achievable by technologies available 
    to the taxi manufacturers. \\ 
    Taxi Companies & Employers of taxi drivers or centralize dispatch 
    operation for taxi drivers	& Solution may require additional funding 
    or approval from the taxi companies for implementation. \\
    City of Toronto & Standardizes and establishes relevant policies 
    affecting taxi drivers \cite{CityofToronto}	& Successful designs or design changes of  
    taxis may require endorsement by the city  \\
    \end{tabular}
\end{table}

\section{Needs}
\label{sec:needs}
\subsection{Back pain}
The elevated risks and levels of low back pain (LBP) that urban taxi drivers 
experience are stimulated by driving. 

Whole body vibration (WBV) defines mechanical energy oscillations transferred to 
the body as a whole [ppt], which taxi drivers experience \cite{KneePain, Serious} 
through the car seat \cite{ppt}. The fact that WBV are the most common and hazardous 
vibratory exposure regarding back problems in occupational life illustrates the
link between the two \cite{ODrivers@Risk}. Also, as low back pain(LBP) is stimulated by 
prolonged sitting \cite{Okunribido2008}, a greater amount time driving, induces a 
higher incidence rate of back injuries\cite{Question?}. This is reinforced by a positive 
correlation between driving a vehicle and patients experiencing LBP \cite{ODrivers@Risk}. 
Other work-related stresses in taxi drivers such as the restrictive environment, bending/twisting 
movements, air pollution, violence, and psychological strains may intensify the risk, 
and furthermore development of LBP \cite{KneePain, POSTULATED}. 

The prevalence of LBP in taxi drivers is reinforced the by statistical evidence that occurrence of LBP 
in taxi and chauffeur drivers is almost double the average in the US \cite{68}. According to a study conducted in Norway, low 
back pain is more prevalent in Norwegian taxi drivers (66\% for women, and 59\% for men) 
compared to the reference population (58\% for women, and 51\% for men). Similar results 
were established for Taiwanese urban taxi drivers \cite{Question?}.

%LBP in taxi drivers is prevalent and induced by the whole body vibrations, prolonged sitting, 
%and a restricted range of motion, all of which are caused by the taxi workplace environment. 
 
\subsection{knee pain}
Taxi drivers experience elevated levels of knee pain, resulting mainly from excessive use 
of the knee joint in order to perform their service. 

Elevated levels of knee pain in taxi drivers is illustrated by its higher occurrence in drivers 
in Norway and Taiwan versus their respective local communities \cite{KneePain}. Repetitive strain 
injury (RSI), defined as an injury stimulated in a certain part of the body due to its overuse, is the 
root of knee pain in drivers \cite{RSI} as they constantly use their knee to operate a vehicle 
\cite{KneePain}. Prior to 1990, 25\% of drivers experienced knee pain compared with 18\% who 
operated vehicles made after 1990 \cite{KneePain}. This decrease coincides with the widespread shift from 
manual and automatic transmission, which requires more repetitive motion in the knee joint 
\cite{KneePain}. This illustrates that a greater amount of pedal-pressing, and by extension to 
driving, aggravates and elevates the risk of knee pain. Data suggesting that “prolonged or repeated 
knee bending is a risk factor for knee OA [osteoarthritis], and that risk may be higher in jobs which 
entail both knee bending and mechanical loading” \cite{Osteoarthritis} supports the correlation 
between driving and knee pain.


%It is evident that greater amounts of driving, which provoke repetitive strain injuries, induces knee pain.

\subsection{Neck}
Prolonged sitting and twisting of the trunk, requirements of the taxi driver occupation \cite{neck}, 
directly stimulate neck pain \cite{neck}. 

The Canadian Centre for Occupational Health and Safety believes that driving can cause pain in the 
neck\cite{proof}. This is supported by correlations between between both prolonged sitting and neck pain as well as  
twisting or bending of the trunk and neck pain \cite{neck}. 
As a taxi driver experiences prolonged sitting periods and frequently bend/twist \cite{Okunribido2008, POSTUALTED},
it is evident that their work-place environment stimulates neck pain. 

\section{Reference Designs}
\label{sec:designs}

This section presents developed, proposed solutions that deal with the issue of
chronic pain in similar communities. Such communities are the truck and bus driver communities, as well as occupations 
that require extended and frequent sitting periods. The solutions are subsected into
redesigned seats, minimizing vibration, and redesigning the cab (driver area) to make operating a vehicle
less stressful on the lower back of the body.
%need to argue why these communities are comparable to the taxi driver community

\subsection{Problems with sitting}
\label{sec:sittingproblems}

%I'm not sure where this is meant to go, but it should go somewhere):**!!!!
Sitting for extended periods of time is a harmful to the human body. As an individuals's weight is adjusted
onto the thighs, it compresses the main arteries beneath the thighs, decreasing 
blood flow to the lower limbs. As thigh muscles perform work to help maintain an upright posture, 
lactic acid forms due to the restriction of oxygen to the area. A build-up of lactic acid results in muscle spasms, also known as static
loading, and eventual muscle fatigue. Furthermore, all weight above the waist is balanced on the ischial 
tuberosities (in the buttocks), which is like an inverted pyramid, all the weight is 
fixed on a centralized point, leading to discomfort. Moreover, the geometry of most seats encourages a flattening of the 
spine. The lower spine curves inwards towards the stomach in an equilibrium position. When the natural curvature 
is flattened, the lumbar discs of the spine compress together, initiating a migration of 
the gel insulating each disc. In the absence of cushioning gel, degradation of the disc occurs, which 
ultimately results in aggravation of the nerves (back pain), hernia and eventual rupture. The latter two cases
require surgery\cite{ergoCentricAnatomy2011}.

\subsection{What constitutes a good seat?}
A good seat includes a pneumatic seat adjustment that acts as a shock absorber for the spine. The seat 
allows a person to sit upright, against the backrest with full contact, while still leaving at least a 
three finger width clearance between the back of the knees and the seat. A shorter seat span is more ideal. 
%what does the next sentence mean, waterfall downwards? Unclear. 
The seat should slant downwards at the edge of the seat, known as a waterfall, to reduce pressure on the veins 
beneath the thighs, and curve upwards at the sides to redistribute weight across a larger surface area. 
%explain what dishing up or down is, it's not totally clear
The seat should neither dish 
up or down in the centre, because dishing up results in pressure on the spine and dishing down stimulates pressure on the veins. 
A faint ridge to match the geometry of the buttocks is optimal\cite{ergoCentricchair2011, Natpost2005}.


\subsection{Ergonomic Seat: ObusForme}
The ObusForme Highback Backrest Support is an example of a redesigned ergonomic seat to minimize body pain. 
The seat has firm lumbar support that pushes the spine back into its natural position. 
It also has lateral curves that fit the geometry of the upper back to reduce muscle work\cite{ObusFormebackrest}. 
The backrest angle relative to the seat is just over 90\textdegree. The support also has a neck brace which supports
the neck, keeping the head aligned with the spine\cite{ObusFormedriverchair}.
ObusForme also has driver seats that reduce pressure on the veins and redistributes weight 
on both the spine and the ischial tuberosities by shaping to the human body\cite{ObusFormecushion}.
\begin{figure}[h]
  \centering
  \includegraphics[scale=0.1]{hb-blk-ca.jpg}
  \caption{ObusForme ergonomic seat}
\end{figure}
\subsection{Minimizing vibration: Bose ride system}
Truck drivers travel up to 12 hours per day, and as a result, face many of the same issues as taxi drivers. 
The prevelent issue is vibration and its effect on the spine. The Bose seat actively and effectively minimizes vibration 
by detecting vertical motion as small as 0.25mm displacements. %placed later: This reduces stress on the spine. 
The detector pulses at 1000 Hz and the computer process the 
wave information and responds with a signal to an actuator that operates a pneumatic pump to raise or lower the seat to 
counter the incoming shock. The seat lowers vibrations to between 0.2-0.3 eVDV (estimated vibration dose value); normal values range 
between 0.8-1.2 eVDV\cite{Bosetruckseat}. A reduced magnitude of vibration dimishes stress on the spine. The Bose ride system shapes to back in the same way as the 
ObusForme car seat\cite{Bosetruckseat}.
\begin{figure}[h]
  \centering
  \includegraphics[scale=0.5]{brt_drivervalue_main.png}
  \caption{Bose vibration reduction system for truck seats}
\end{figure}
\subsection{Ergonomic work environment: Bus driver cab}
The following report was compiled by the Optimal Performance Consultants company regarding the 
inadequacies of the operator's area of a bus and the negative effects on the health of the driver who, similar to a taxi driver, has to sit 
for many hours a day\cite{Ismail2003}. In addition to back pain, the report 
identified chronic wrist and ankle pain due to pedal and steering operation. Among suggestions to modify the 
seat to be more ergonomic, the report suggested changes to enhance the ergonomics of the work envroment. 
The suggestions were as follows:
\begin{enumerate}
\item Pedals should not require a rotation of more than 25 degrees relative to the rest position. 
\item For more frequent operation of the pedal, the minimum force limit should be smaller, 
i.e. more sensitive pedals. Minimum lower bound is 40N of force, and the supremum is 400N.
\item The pedals should be mounted on an incline as opposed to flat against the floor to enable the leg 
muscles to exert most of the force, minimizing strain on the ankles. 
\item Utilizing pedals, instead of a wrist operation, to signal lane changing would remove the wrist operation
and avoid RSI in the wrist.
%\item Most important was the suggestion to implement pedals to signal change of lane, thereby removing 
%the wrist operation procedure and wear on the wrists\cite{Ismail2003}. 
\end{enumerate}



\section{Engineering Problems}
\label{sec:engprob}
Several solutions regarding the issues of chronic pain have been presented
(see section~\ref{sec:designs}), but implementing these solution in the context
of a taxi raises problems, which will be examined in this section. These 
problems are space, ease of use, and installation.
%sources for the entire following section?

\subsection{Space}
\label{sec:space}

Solutions exist that are designed to function as desk chairs or seats for truck
drivers. These solutions, however, would need to be altered to fit in
a taxi. The CVG GSX 3000 truck seat, for example, measures 1113~mm from base
to top, and is 522~mm wide. The dimensions of a mid-size automobile
are more restrictive than a truck, so solutions will have to deal with
this reduced space.

\subsection{Ease of Use}
\label{sec:ease}
Taxi drivers frequently enter and exit their vehicles in the course of
their duties, whether to assist customers with bags, or simply while awaiting
a customer. Compared to long-haul truck drivers, taxi drivers must have less
difficulty entering and exiting their vehicles. This should be kept in mind
while designing any aspect of the driver's area.
\subsection{Installation}
\label{sec:installation}

The vast majority of taxis are modified production cars, % Source: Interviews?
and so certain aspects of existing reference designs would be difficult to
implement. The standard steering wheel of an automobile, for example,
would be difficult to modify to a 60\textdegree~range of
rotation\cite{Ismail2003}. Feasibility of installation must then
be considered when assessing any implementation of a solution.

\section{Design Philosophy}
\label{sec:philo}
The design philosophy will guide the beginning stages of developing a
solution for the health needs of Taxi Drivers in the City of
Toronto. There are three key principles that should
guide the design process of the solution to the outlined problem.
 
\subsection{Efficacy}
The design must be successful in improving the quality of life of taxi 
drivers in the city of Toronto, in particular, alleviating chronic pain 
that taxi drivers experience. The main reason taxi drivers consider a 
career change is health concerns \cite{chicago}. Thus, it is crucial that 
any design focuses on efficacy as the main priority. The degree to which 
the persistent pain of taxi drivers is reduced is the indicator of 
effectiveness. The ultimate goal would be to completely eliminate pain 
and create comfort for taxi drivers in a working environment. 

\subsection{Prevention}
In addition to providing relief to drivers currently suffering from 
chronic pain, an ideal design should also seek to prevent future long-term 
health problems. For new drivers, a preventative design would eliminate
the risk of chronic pain due to the occupation, while for veteran 
drivers, prevention entails assurance that the chronic pain reduced does
not return. 
 
\subsection{Accessibility}
% I don't know if accessibility is the best title for this section,
% since it suggests that it's about accommodating people with disabilities
% instead of reducing cost
Due to the unique needs of the Taxi driver community, accessibility is 
important to accommodate. Health is a concern of all drivers in the Taxi 
industry and thus an effective design should be accessible by all taxi 
drivers in the City of Toronto. Considering that a majority of taxi 
drivers experience economic uncertainty\cite{facey2003health}, any design 
successful in improving the quality of life in taxi drivers must be
economically accessible by all taxi drivers or the operating 
budgets of the major taxi companies in the city of Toronto. 
 
\subsection{Convenience}
For Taxi drivers, every minute could translate into possible business 
time\cite{ facey2003health}. The time consumed by taxi drivers to operate 
any additional device should not deter from their incomes. The demographic 
trends of taxi drivers need to be consider in order to create a design 
that enables all drivers to achieve an elevated level of health 
without additional time consuming efforts.

\section{Project Requirements}
\label{sec:requirements}
This section will present the objectives, constraints, and criteria 
that frame the problem.

\subsection{High-level Objectives}
\label{sec:high-level-objectives}

\begin{enumerate}
\item Reduce the \emph{frequency} of chronic pain among Toronto taxi drivers
\item Reduce the \emph{severity} of chronic pain among Toronto taxi drivers
\item Achieve both of the above at a minimum cost to the stakeholders
\end{enumerate}

\subsection{Constraints}
\begin{enumerate}
\item \emph{Must} be capable of being installed in existing taxis.
\item \emph{Must} comply with all of the Canadian Motor Vehicle Safety Regulations,
specifically Schedule~IV \cite{motorregs}.
\item \emph{Must not} reduce the amount of available passenger space.
\item \emph{Must not} hinder ingress and egress from the vehicle.
\end{enumerate}
\subsection{Criteria}
The following criteria will be used to assess all designs.
\begin{table}[h]
\centering
\caption{Criteria and corresponding metrics}
\begin{tabular}{c p{10cm} }
  Criteria & Metric \\ \hline
  Cost & Combined manufacturing and installation price (lower is better) \\
  Ease of installation & Time required to install in taxi (less is better) \\
  Breadth of effectiveness & Percentile range accommodated by device (higher is better) \\
  Durability & Time before required maintenance (more is better) \\
  Severity & Level of reported pain according to test (lower is better) \\
  Frequency & Number of drivers reporting chronic pain (lower is better)
\end{tabular}
\end{table}

\subsection{Summary}

\begin{table}[h!]
\centering
\caption{Summary of Objectives devolving constraints and criteria}
\begin{tabular}{l l p{5cm}}
Objective & Corresponding constraint/criteria & Corresponding metric \\ \hline
\multirow{2}{*}{Reduce Frequency} & Frequency & Number of drivers reporting pain \\
& Breadth of effectiveness & Percentile range accommodated \\
Reduce Severity & Severity & Reported pain level \\
\multirow{3}{*}{Minimize Cost} & Unit Cost & Combined manufacturing and installation price \\
& Ease of Installation & Time required to install \\
& Durability & Time before required maintenance \\
\end{tabular}
\end{table}
\bibliographystyle{IEEEtranN}
\bibliography{references,sourcessam,sourcesjudy,sourcesphil,sourcessam2,sourcesphil2}

\end{document}

